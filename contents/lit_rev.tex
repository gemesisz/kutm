\section{Related Work}
\label{ch:lit_rev}

To start off our research we studied previously published papers on the topic and extended our initial idea based on their findings. At the end of each article review, we explain what significance the paper has to our research.

\subsection{Cheating and Detection Method in Massively Multiplayer Online Role-Playing Game: Systematic Literature Review \cite{9766355}}

The main topic of this review is the examination of possible cheat in MMORPG (Massively Multiplayer Online Role-Playing Game) games, the exploration of its causes, introduction of its types and presentation of a possible detection method.
In the reaserch, they discussed the structures of MMORPG games and described the types of cheating behaviours.

The authors show countermeasures to prevent cheating which comes from the structure of server and from the player's behavior.
They classified the detecting methods into five groups: statistics analysis, data mining, similarity analysis, and network-based analysis and matched with the cheating behaviours.
Finally, they showed how they can detect cheaters with machine learning algorithms.

This paper provides extensive research on various occurrence cheating methods and solutions for detecting cheaters in MMORPGs.

Based on the findings of this paper we learned about server side cheat prevention and broadened our knowlegde on the cheating habits of MMORPG players, this showed us that different types of games have different cheater habits. We added several types of analysis to our method in order to detect the cheating habits and widen the coverage of our anti-cheat system. 

In this paper the researchers focused on the aspects of cheating mainly in MMORPG games. Compared to this, in our research we wanted to expand our cheat detection model to different video games. Our main focus was on first-person shooters but our intention with the research was to create an anti-cheat system that is applicable in almost any type of video game.

\subsection{Robust Vision-Based Cheat Detection in Competitive Gaming \cite{jonnalagadda2021robust}}

In the paper “Robust Vision-Based Cheat Detection in Competitive Gaming” we can explore an
already existing solution that tries to tackle the cheating problem in Counter Strike - Global
Offensive, and another first-person shooter game. The main idea of this method of cheat
detection is using DNNs (deep neural networks) to look for rendered frames, that give away
visual evidence of a cheating software being used. This can be information in the game that the
player should not have (seeing otherwise invisible enemies through walls, or on the map), or the
GUI (graphical user interface) of the cheat itself.

The way it is implemented is very straightforward, yet novel. The GPU’s frame buffer is captured
right before it appears on the computer screen, and is analyzed by a trained DNN. The DNN is
trained to look for altered and suspicious pixel patterns. The model's training took place on a
large set of frames from the games with and without cheats. It calculates both the probability of
a frame containing cheats and the confidence value, which is used both to avoid false positives
and to help the developers know when the model needs to be retrained, for example when a
cheat software got changed, to avoid detection.

The results were auspicious: The more visually prominent the cheating was, the better the
system got at successfully detecting the presence of cheats. Different configurations were
mentioned, and the exact results varied based on the configuration in use. The highest overall
accuracy measured was 0.89.
The developers noted, that this implementation should be used as an extension of already
existing cheat detection methods. Overall in my opinion it is a great way to deal with cheaters in
online games, and it is worth exploring further when it comes to AI cheat detection.

This paper provided us with in-depth information about the use of DNNs in anti-cheat systems and important ways of detecting cheaters, so we based our use of DNNs on the findings of the article which improved our initial results.

Compared to this paper's findings about image-based cheat detection we added more types of cheat prediction methods with the help of CNNs in addition to DNNs. Further expanding the abilities of our anti-cheat system, we included XAI, which helped us raise the prediction rate compared to the highest achievable rate using the findings of their research.

\subsection{Deep learning and multivariate time series for cheat detection in video games \cite{pinto2021deep}}

This research introduces a novel cheat detection system for online video games that, instead of traditional in-game data, is based on human-computer interaction (HCI) data, such as keystrokes and mouse movements. Existing systems use game-specific customization to detect cheating, while this approach uses player interaction data as multivariate time series analyzed by convolutional neural networks (CNNs). This makes the system adaptable to various games and capable of detecting cheaters without any game-specific information. The method successfully identifies cheats like aimbots and triggerbots in first-person shooter (FPS) games by recognizing behavioral patterns that differ from typical human interactions.

The study was performed on data collected from players of Counter-Strike: Global Offensive, achieving detection rates of 99.2\% for triggerbots and 98.9\% for aimbots. Cross-validation was used to ensure the model’s robustness and ability to adapt to new player inputs, demonstrating its effectiveness in real-world scenarios. The use of universal input data makes this system, compared to other systems, a flexible and scalable solution for cheat detection. This makes it easy to apply to other games without the need for any specific tuning. This research provides significant advancements in cheat detection methodologies by a game-independent approach that can also adapt to new and evolving forms of cheating.

This article shined a light on the importance of checking input data and human interactions, none of which we have considered initially in our system. Examining human interactions creates a more reliable way of cheat-detection so we implemented it in our method.

While this research also applied CNNs to detect cheating behaviour, in our research we tried to broaden the amount of video games the anti-cheat system is applicable to, by including more AI methods to our system. In comparison, the prediction rates achieved in this paper exceed ours, but their research focused strictly on first-person shooters and cheats specific to those video games, while our research targets a much wider range of games.

\subsection{Explainable AI for Cheating Detection and Churn Prediction in Online Games \cite{tao2022explainable}}

The paper explores the application of AI and XAI (eXplainable AI) in online gaming, focusing on issues such as game cheating detection and player churn prediction. It discusses how AI has historically been applied in games for tasks like matchmaking and behaviour monitoring. XAI specifically addresses the black-box nature of AI models, enhancing trust by providing explanations for AI decisions. The authors introduce datasets from NetEase Games, using a large amount of game logs, player behaviour sequences, client images, and social graphs to improve the transparency and accuracy of AI models.

The authors concluded that the use of XAI could be preferred over the black-box model having received more positive feedback from game operators and game designers alike for its’ transparency in reasoning and explanation. These results are far better than the authors expected and they claim that this technology should be further examined and developed to achieve even better results.

The findings of this paper proved to us that having explanations for the AI models' behaviour and reasoning is just as important as detecting the cheats themselves. Without an explanation we can't really know why certain actions are detected as cheating, with an explanation we reduce the amount of mistakes the cheat-detection system could make.

Their approach integrates XAI for churn prediction alongside cheat detection in multiple video games, much like our intentions. Expanding on the ideas of this paper, our main focus shifted more towards the cheat detection. We added more AI methods to our anti-cheat system, helping us detect cheaters more efficiently, while still maintaining the transparency available with XAI.