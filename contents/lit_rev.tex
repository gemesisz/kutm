\section{Related Work}
\label{ch:lit_rev}

\subsection{Cheating and Detection Method in Massively Multiplayer Online Role-Playing Game: Systematic Literature Review \cite{9766355}}

The main topic of this review is the examination of possible cheat in MMORPG games, the exploration of its causes, introduction of its types and presentation of a possible detection method.
In the reaserch, they discussed the structures of MMORPG games and described the types of cheating behaviours.

The authors show countermeasures to prevent cheating which comes from the structure of server and from the player's behavior.
They classified the detecting methods into five groups: statistics analysis, data mining, similarity analysis, and network-based analysis and matched with the cheating behaviours.
Finally, they showed that how they can detect cheaters with machine learning algorithms.

This paper provides extensive research on various occurrence cheating methods and solutions for detecting cheaters in MMORPGs.

\subsection{Robust Vision-Based Cheat Detection in Competitive Gaming \cite{jonnalagadda2021robust}}

In the paper “Robust Vision-Based Cheat Detection in Competitive Gaming” we can explore an
already existing solution that tries to tackle the cheating problem in Counter Strike - Global
Offensive, and another first-person shooter game. The main idea of this method of cheat
detection is using DNNs (deep neural networks) to look for rendered frames, that give away
visual evidence of a cheating software being used. This can be information in the game that the
player should not have (seeing otherwise invisible enemies through walls, or on the map), or the
GUI (graphical user interface) of the cheat itself. \newline
The way it is implemented is very straightforward, yet novel. The GPU’s frame buffer is captured
right before it appears on the computer screen, and is analyzed by a trained DNN. The DNN is
trained to look for altered and suspicious pixel patterns. The model's training took place on a
large set of frames from the games with and without cheats. It calculates both the probability of
a frame containing cheats and the confidence value, which is used both to avoid false positives
and to help the developers know when the model needs to be retrained, for example when a
cheat software got changed, to avoid detection. \newline
The results were auspicious: The more visually prominent the cheating was, the better the
system got at successfully detecting the presence of cheats. Different configurations were
mentioned, and the exact results varied based on the configuration in use. The highest overall
accuracy measured was 0.89.
The developers noted, that this implementation should be used as an extension of already
existing cheat detection methods. Overall in my opinion it is a great way to deal with cheaters in
online games, and it is worth exploring further when it comes to AI cheat detection.

\subsection{Deep learning and multivariate time series for cheat detection in video games \cite{pinto2021deep}}

This research introduces a novel cheat detection system for online video games that, instead of traditional in-game data, is based on human-computer interaction (HCI) data, such as keystrokes and mouse movements. Existing systems use game-specific customization to detect cheating, while this approach uses player interaction data as multivariate time series analyzed by convolutional neural networks (CNNs). This makes the system adaptable to various games and capable of detecting cheaters without any game-specific information. The method successfully identifies cheats like aimbots and triggerbots in first-person shooter (FPS) games by recognizing behavioral patterns that differ from typical human interactions.

The study was performed on data collected from players of Counter-Strike: Global Offensive, achieving detection rates of 99.2\% for triggerbots and 98.9\% for aimbots. Cross-validation was used to ensure the model’s robustness and ability to adapt to new player inputs, demonstrating its effectiveness in real-world scenarios. The use of universal input data makes this system, compared to other systems, a flexible and scalable solution for cheat detection. This makes it easy to apply to other games without the need for any specific tuning. This research provides significant advancements in cheat detection methodologies by a game-independent approach that can also adapt to new and evolving forms of cheating.

\subsection{Explainable AI for Cheating Detection and Churn Prediction in Online Games \cite{tao2022explainable}}

The paper explores the application of AI and XAI (eXplainable AI) in online gaming, focusing on issues such as game cheating detection and player churn prediction. It discusses how AI has historically been applied in games for tasks like matchmaking and behaviour monitoring. XAI specifically addresses the black-box nature of AI models, enhancing trust by providing explanations for AI decisions. The authors introduce datasets from NetEase Games, using a large amount of game logs, player behaviour sequences, client images, and social graphs to improve the transparency and accuracy of AI models.

The authors concluded that the use of XAI could be preferred over the black-box model having received more positive feedback from game operators and game designers alike for its’ transparency in reasoning and explanation. These results are far better than the authors expected and they claim that this technology should be further examined and developed to achieve even better results.
